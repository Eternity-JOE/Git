\documentclass[a4paper, 12pt]{article}
\usepackage[UTF8]{ctex}
\usepackage{graphicx}
\usepackage{listings} % 引入代码高亮包
\usepackage{geometry} % 页面边距设置
\geometry{a4paper, margin=1in}
\usepackage{fancyhdr} % 页眉和页脚设置
\usepackage{hyperref} % 插入超链接
\usepackage{tcolorbox} % 创建框架
\usepackage{pgfplots} % 绘制图形
\usepackage{media9} % 插入视频
\usepackage{xcolor} % 文本颜色
\usepackage{soul} % 高亮文本
\usepackage{float}



\begin{document}

\begin{titlepage}
 \centering
    \includegraphics[width=0.5\linewidth]{1.png}
    
    \vspace{-1.5em} % 减少两张图片之间的垂直间距
    
    \includegraphics[width=0.5\linewidth]{中文全称横式2.png}
    
    \Huge
    \textbf{命令行环境与Python实验报告} % 标题
    
    \vspace{1in} % 标题与作者之间的间距
    
    \Large
    \songti 
    姓名:乔宇恒
    \\学号:23030021059
    
    \vspace{1in} % 作者与日期之间的间距
    
    \Large
    \today 
    
    \vfill % 使下面的内容向页面底部对齐

    \thispagestyle{empty} % 去除封面页的页码
\end{titlepage}

\pagenumbering{arabic}
\tableofcontents
\newpage

\section{命令行环境}

\subsection{使用 pgrep 来查找 pid}
\begin{figure}[H]
    \centering
    \includegraphics[width=0.5\linewidth]{屏幕截图 2024-09-11 205304.png}
    \caption{使用 pgrep 来查找 pid}
\end{figure}

\subsection{使用 pkill 结束进程而不需要手动输入 pid}
\begin{figure}[H]
    \centering
    \includegraphics[width=0.5\textwidth]{屏幕截图 2024-09-11 205304.png}
    \caption{使用 pkill 结束进程而不需要手动输入 pid}

\end{figure}

\subsection{wait 命令}
\begin{figure}[H]
    \centering
    \includegraphics[width=0.5\textwidth]{屏幕截图 2024-09-11 210014.png}
    \caption{wait 命令}
\end{figure}

\subsection{编写一个 bash 函数 pidwait}
\begin{lstlisting}
pidwait() {
    local pid="$1"

    # 检查 PID 是否为空
    if [ -z "$pid" ]; then
        echo "Usage: pidwait <pid>"
        return 1
    fi

    # 一直循环,直到进程结束
    while kill -0 "$pid" 2>/dev/null; do
        sleep 1  # 每隔 1 秒检查一次进程状态
    done

    echo "Process $pid has ended."
}

\end{lstlisting}

\subsection{创建别名}
\begin{figure}[H]
    \centering
    \includegraphics[width=0.5\textwidth]{屏幕截图 2024-09-11 210617.png}
    \caption{创建别名}
\end{figure}

\subsection{别名语法}
\begin{lstlisting}
    alias alias_name="command_to_alias arg1 arg2"
\end{lstlisting}

\subsection{文件操作}
\begin{lstlisting}
列出目录内容:ls
切换目录:cd /path/to/directory
查看当前目录:pwd
创建目录:mkdir directory_name
删除目录:rmdir directory_name 或 rm -r directory_name(删除非空目录)
删除文件:rm file_name
复制文件:cp source_file destination
移动/重命名文件:mv source_file destination
\end{lstlisting}

\subsection{文件内容查看}
\begin{lstlisting}
查看文件内容:cat file_name
分页查看文件内容:less file_name 或 more file_name
显示文件头部:head file_name
显示文件尾部:tail file_name
\end{lstlisting}

\subsection{系统信息}
\begin{lstlisting}
查看磁盘使用情况:df -h
查看内存使用情况:free -h
查看 CPU 信息:lscpu
查看系统版本:uname -a
\end{lstlisting}

\subsection{进程管理}
\begin{lstlisting}
查看进程:ps aux 或 top
杀死进程:kill process_id
强制杀死进程:kill -9 process_id
查找进程:pgrep process_name
\end{lstlisting}

\subsection{Windows系统中的文件和目录操作}
\begin{lstlisting}
列出目录内容:dir
切换目录:cd path\to\directory
查看当前目录:cd 或 echo %cd%
创建目录:mkdir directory_name
删除目录:rmdir directory_name 或 rd directory_name
(删除非空目录使用 /s 选项)
删除文件:del file_name
复制文件:copy source_file destination
移动/重命名文件:move source_file destination
\end{lstlisting}

\subsection{Windows系统中的进程管理}
\begin{lstlisting}
查看进程:tasklist
杀死进程:taskkill /PID process_id
强制杀死进程:taskkill /F /PID process_id
查找进程:tasklist | findstr process_name
\end{lstlisting}

\newpage
\section{Python}
\subsection{Python简介}
Python 是一种广泛使用的高级编程语言,以其简洁、易读的语法和强大的功能著称。它支持多种编程范式,包括面向对象、过程式和函数式编程,并且提供了丰富的标准库和第三方模块,使得开发各种应用程序变得高效且灵活。作为一种跨平台的解释型语言,Python 被广泛应用于数据分析、机器学习、Web 开发、自动化脚本等多个领域。


\subsection{Python 基础操作}
\subsubsection{正则表达式}
\begin{lstlisting}
import re

result = re.match(r'\d+', '123abc')
if result:
    print(result.group())  # 输出: 123
\end{lstlisting}
从字符串的起始位置匹配正则表达式模式,如果匹配成功,返回一个匹配对象;否则返回 None。
\subsubsection{变量赋值}
\begin{lstlisting}
x = 10
y = 20
z = x + y
print(z)  # 输出: 30
\end{lstlisting}

\subsubsection{条件语句}
\begin{lstlisting}
x = 5
if x > 0:
    print("x 是正数")
elif x == 0:
    print("x 是零")
else:
    print("x 是负数")
\end{lstlisting}

\subsubsection{循环}
\begin{lstlisting}
for i in range(5):
    print(i)  # 输出: 0 1 2 3 4
\end{lstlisting}

\subsubsection{列表操作}
\begin{lstlisting}
numbers = [1, 2, 3, 4, 5]
print(numbers[0])  # 输出: 1
numbers.append(6)
print(numbers)  # 输出: [1, 2, 3, 4, 5, 6]
\end{lstlisting}

\subsubsection{字典操作}
\begin{lstlisting}
person = {'name': 'Alice', 'age': 25}
print(person['name'])  # 输出: Alice
person['age'] = 26
print(person)  # 输出: {'name': 'Alice', 'age': 26}
\end{lstlisting}

\subsubsection{函数定义}
\begin{lstlisting}
def add(x, y):
    return x + y

result = add(5, 3)
print(result)  # 输出: 8
\end{lstlisting}

\subsubsection{列表推导式}
\begin{lstlisting}
numbers = [1, 2, 3, 4, 5]
squares = [n**2 for n in numbers]
print(squares)  # 输出: [1, 4, 9, 16, 25]
\end{lstlisting}

\subsubsection{文件读取}
\begin{lstlisting}
with open('example.txt', 'r') as file:
    content = file.read()
    print(content)
\end{lstlisting}

\subsubsection{异常处理}
\begin{lstlisting}
try:
    x = int(input("输入一个数字: "))
    print(x)
except ValueError:
    print("输入无效,请输入数字")
\end{lstlisting}

\subsubsection{类与对象}
\begin{lstlisting}
class Person:
    def __init__(self, name, age):
        self.name = name
        self.age = age

    def greet(self):
        print(f"Hello, my name is {self.name}")

p = Person('Alice', 25)
p.greet()  # 输出: Hello, my name is Alice
\end{lstlisting}

\subsubsection{列表切片}
\begin{lstlisting}
numbers = [1, 2, 3, 4, 5]
print(numbers[1:4])  # 输出: [2, 3, 4]
\end{lstlisting}



\subsubsection{集合操作}
\begin{lstlisting}
numbers = {1, 2, 3, 4}
numbers.add(5)
print(numbers)  # 输出: {1, 2, 3, 4, 5}
\end{lstlisting}

\subsubsection{迭代器}
\begin{lstlisting}
my_list = [1, 2, 3]
iterator = iter(my_list)
print(next(iterator))  # 输出: 1
print(next(iterator))  # 输出: 2
\end{lstlisting}

\subsubsection{列表排序}
\begin{lstlisting}
numbers = [5, 3, 1, 4, 2]
sorted_numbers = sorted(numbers)
print(sorted_numbers)  # 输出: [1, 2, 3, 4, 5]
\end{lstlisting}

\subsection{Python视觉应用}
\begin{lstlisting}
from PIL import Image, ImageFilter

# 打开图像
image_path = "input_image.jpg"
image = Image.open(image_path)

# 1. 显示原始图像
image.show()

# 2. 灰度化处理
gray_image = image.convert("L")
gray_image.show()

# 3. 图像缩放
resized_image = image.resize((image.width // 2, image.height // 2))  # 将图像缩小一半
resized_image.show()

# 4. 边缘检测
edges_image = image.filter(ImageFilter.FIND_EDGES)
edges_image.show()

# 5. 保存处理后的图像
gray_image.save("output_gray_image.jpg")
resized_image.save("output_resized_image.jpg")
edges_image.save("output_edges_image.jpg")

print("图像处理完成,处理后的图像已保存。")

\end{lstlisting}


\newpage
\section{实验感悟}
学习了命令行环境的基本操作,了解了Python语言入门基础知识和在视觉方面的应用
\section{Github网址}
\href{https://github.com/Eternity-JOE/Git}{https://github.com/Eternity-JOE/Git}

\end{document}
