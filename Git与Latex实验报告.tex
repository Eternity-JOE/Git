\documentclass[a4paper, 12pt]{article}
\usepackage[UTF8]{ctex}
\usepackage{graphicx}
\usepackage{listings} % 引入代码高亮包
\usepackage{geometry} % 页面边距设置
\geometry{a4paper, margin=1in}
\usepackage{fancyhdr} % 页眉和页脚设置
\usepackage{hyperref} % 插入超链接
\usepackage{tcolorbox} % 创建框架
\usepackage{pgfplots} % 绘制图形
\usepackage{media9} % 插入视频
\usepackage{xcolor} % 文本颜色
\usepackage{soul} % 高亮文本

\title{Git与Latex实验报告}
\author{乔宇恒}
\date{\today}

\begin{document}

\begin{titlepage}
    \centering
    \begin{figure}
        \centering
        \includegraphics[width=0.5\linewidth]{1.png}
    \end{figure}
    \vspace*{1in} % 顶部空白
    
    \Huge
    \textbf{Git与Latex实验报告} % 标题
    
    \vspace{1in} % 标题与作者之间的间距
    
    \Large
    \songti 
    姓名:乔宇恒
    \\学号:23030021059
    
    \vspace{1in} % 作者与日期之间的间距
    
    \Large
    \today 
    
    \vfill % 使下面的内容向页面底部对齐

    \thispagestyle{empty} % 去除封面页的页码
\end{titlepage}

\pagenumbering{arabic}
\tableofcontents
\newpage

\section{Git}
\subsection{初始化Git仓库}
\begin{figure}[h!]
    \centering
    \includegraphics[width=1\linewidth]{屏幕截图 2024-08-30 004611.png}
\end{figure}

\subsection{克隆一个现有的 Git 仓库}
\begin{figure}[h!]
    \centering
    \includegraphics[width=1\textwidth]{屏幕截图 2024-08-30 005446.png}
\end{figure}

\subsection{检查当前仓库状态}
\begin{figure}[h!]
    \centering
    \includegraphics[width=1\textwidth]{屏幕截图 2024-08-30 005826.png}
\end{figure}

\subsection{添加文件到暂存区}
\begin{figure}[h!]
    \centering
    \includegraphics[width=1\textwidth]{屏幕截图 2024-08-30 010043.png}
\end{figure}

\subsection{添加 GitHub 远程仓库}
\begin{figure}[h!]
    \centering
    \includegraphics[width=1\textwidth]{屏幕截图 2024-08-30 011821.png}
\end{figure}

\subsection{检查远程仓库是否添加成功}
\begin{figure}[h!]
    \centering
    \includegraphics[width=1\textwidth]{屏幕截图 2024-08-30 012028.png}
\end{figure}

\subsection{提交暂存区的更改}
\begin{lstlisting}
git commit -m "Commit message"
\end{lstlisting}

\subsection{查看提交历史}
\begin{lstlisting}
git log 
\end{lstlisting}

\subsection{查看当前分支}
\begin{lstlisting}
git branch
\end{lstlisting}

\subsection{创建新分支}
\begin{lstlisting}
git branch <branch name>
\end{lstlisting}

\subsection{切换到指定分支}
\begin{lstlisting}
git checkout -b <branch name>
\end{lstlisting}

\subsection{查看简洁的提交历史}
\begin{lstlisting}
git log --oneline
\end{lstlisting}

\subsection{查看更改内容}
\begin{lstlisting}
git diff
\end{lstlisting}

\subsection{查看暂存区和工作区的差异}
\begin{lstlisting}
git diff --cached
\end{lstlisting}

\subsection{合并分支}
\begin{lstlisting}
git merge <branch name>
\end{lstlisting}

\subsection{删除本地分支}
\begin{lstlisting}
git branch -d <branch name>
\end{lstlisting}

\subsection{强制删除本地分支}
\begin{lstlisting}
git branch -D <branch name>
\end{lstlisting}

\subsection{查看远程分支}
\begin{lstlisting}
git branch -r
\end{lstlisting}

\subsection{删除远程分支}
\begin{lstlisting}
git push origin --delete <branch name>
\end{lstlisting}

\subsection{查看远程仓库}
\begin{lstlisting}
git remote -v
\end{lstlisting}

\subsection{推送更改到远程仓库}
\begin{lstlisting}
git push origin <branch name>
\end{lstlisting}

\subsection{拉取远程更改}
\begin{lstlisting}
git pull origin <branch name>
\end{lstlisting}

\subsection{从远程仓库拉取所有分支}
\begin{lstlisting}
git fetch --all
\end{lstlisting}

\subsection{查看所有标签}
\begin{lstlisting}
git tag
\end{lstlisting}

\newpage
\section{LaTeX 基础操作}
\subsection{文档开始和结束}
\begin{lstlisting}[language=TeX]
\documentclass{article}
\begin{document}
% 内容
\end{document}
\end{lstlisting}

\subsection{标题}
\begin{lstlisting}[language=TeX]
\title{标题}
\author{作者}
\date{\today}
\maketitle
\end{lstlisting}

\subsection{生成目录}
\begin{lstlisting}[language=TeX]
\tableofcontents
\end{lstlisting}

\subsection{章节和节}
\begin{lstlisting}[language=TeX]
\section{章节标题}
\subsection{子节标题}
\subsubsection{小节标题}
\end{lstlisting}

\subsection{加粗文本}
\begin{lstlisting}[language=TeX]
\textbf{粗体文本}
\end{lstlisting}

\subsection{斜体文本}
\begin{lstlisting}[language=TeX]
\textit{斜体文本}
\end{lstlisting}

\subsection{下划线}
\begin{lstlisting}[language=TeX]
\underline{下划线文本}
\end{lstlisting}

\subsection{高亮文本}
\begin{lstlisting}[language=TeX]
\usepackage{soul}
\hl{高亮文本}
\end{lstlisting}

\subsection{文本颜色}
\begin{lstlisting}[language=TeX]
\usepackage{xcolor}
\textcolor{red}{红色文本}
\end{lstlisting}




\subsection{插入图片}
\begin{lstlisting}[language=TeX]
\usepackage{graphicx}
\begin{figure}[h!]
\centering
\includegraphics[width=0.5\textwidth]{image.png}
\caption{图片标题}
\label{fig:example}
\end{figure}
\end{lstlisting}

\subsection{调整图片大小}
\begin{lstlisting}[language=TeX]
\includegraphics[width=0.3\textwidth]{image.png}
\end{lstlisting}

\subsection{行内公式}
\begin{lstlisting}[language=TeX]
这是一个行内公式 $a^2 + b^2 = c^2$。
\end{lstlisting}

\subsection{独立公式}
\begin{lstlisting}[language=TeX]
\begin{equation}
E = mc^2
\end{equation}
\end{lstlisting}


\subsection{插入脚注}
\begin{lstlisting}[language=TeX]
这是一个包含脚注的句子\footnote{这是脚注内容}。
\end{lstlisting}

\subsection{引用文献}
\begin{lstlisting}[language=TeX]
\usepackage{cite}
\cite{reference_label}
\end{lstlisting}

\subsection{生成参考文献列表}
\begin{lstlisting}[language=TeX]
\bibliographystyle{plain}
\bibliography{references}
\end{lstlisting}



\subsection{设置页面边距}
\begin{lstlisting}[language=TeX]
\usepackage[a4paper, margin=1in]{geometry}
\end{lstlisting}







\subsection{页眉和页脚}
\begin{lstlisting}[language=TeX]
\usepackage{fancyhdr}
\pagestyle{fancy}
\fancyhead[L]{左页眉}
\fancyhead[C]{中间页眉}
\fancyhead[R]{右页眉}
\end{lstlisting}

\subsection{分页}
\begin{lstlisting}[language=TeX]
\newpage
\end{lstlisting}



\subsection{注释}
\begin{lstlisting}[language=TeX]
% 这是一个注释
\end{lstlisting}



\subsection{文本换行}
\begin{lstlisting}[language=TeX]
这是一个很长的文本,\newline
使用 \texttt{\\} 进行换行。
\end{lstlisting}

\subsection{引用}
\begin{lstlisting}[language=TeX]
\begin{quote}
这是一个引用的段落。
\end{quote}
\end{lstlisting}

\subsection{代码块}
\begin{lstlisting}[language=TeX]
\begin{verbatim}
这是一个代码块
\end{verbatim}
\end{lstlisting}




\newpage
\section{Github网址}
\href{https://github.com/Eternity-JOE/Git}{https://github.com/Eternity-JOE/Git}

\end{document}
